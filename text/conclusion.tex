Zusammenfassend kann gesagt werden, dass die Entwicklung eines Open Source Produkts eine komplexe Aufgabe darstellt. Die Entwickler sitzen nicht lokal in einer Firma, sondern verteilt über die gesamte Welt und kommunizieren vorwiegend über Mailing-Listen oder Kongresse.\\
Aufgrund dieser Herausforderungen existieren beim KDE Projekt idealerweise Verantwortliche und Leiter aller verschiedenen Module, welche unter Anderem die Softwarequalität sicherstellen sollen. Unterstützt werden sie von zahlreichen Guidelines (z.B. für Entwickler). Diese legen Richtlinien bezüglich Architektur, Stil, Oberfläche und vor allem auch Abnahmekriterien fest.\\
In der Praxis scheinen die verwendeten Prozesse recht ordentlich zu funktionieren. Regelmäßige Releases von KDE Plasma usw. bringen sowohl neue Features als auch viele Bugfixes. Die Verbreitung von KDE spricht auch für Kundenzufriedenheit.\\
Für die Zukunft muss das KDE Team jedoch weiter wachsen, denn einige Verantwortungen sind noch nicht fest vergeben. Dies erschwert es natürlich, weiterhin gute Software zu entwickeln und auch größere neue Features gut umzusetzen.